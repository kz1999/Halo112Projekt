\chapter{Specifikacija programske potpore}
		
	\section{Funkcionalni zahtjevi}
			
			\textbf{\textit{dio 1. revizije}}
			
			
			\noindent \textbf{Dionici:}
			
			\begin{packed_enum}
				
				\item Neregistrirani/neprijavljeni korisnik
				\item Spasioc
				\item Doktor(Spasioc)
				\item Vatrogasac(Spasioc)
				\item Policajac(Spasioc)
				\item Voditelj stanice		
				\item Dispečer
				\item Administrator
				\item Razvojni tim
				
			\end{packed_enum}
			
			\noindent \textbf{Aktori i njihovi funkcionalni zahtjevi:}
			
			
			\begin{packed_enum}
				\item \underline{Neregistrirani/neprijavljeni korisnik može:}
				
				\begin{packed_enum}
					
					\item poslati zahtjev za registraciju s željenom ulogom za
					koju se prijavljuje, a potrebni su korisničko ime, fotografija, lozinka, ime, prezime, broj
					mobitela i email adresa\item  \underbar{Aktor 1 (inicijator) može:}
					\item prijaviti se za ulogu: 112 dispečer, doktor,
					vatrogasac i policajac
					
				\end{packed_enum}
			
				\item \underline{Spasioc može:}
				
				\begin{packed_enum}
					
					\item osvježiti podatak o dostupnosti za akcije (dostupan i spreman za akciju ili nije)
					\item odazvati se na akciju spašavanja
					
				\end{packed_enum}
			
				\item \underline{Doktor(Spasioc) može:}
				
				\begin{packed_enum}
					
					\item biti osposobljen za vožnju motociklom ili kao putnik u kolima hitne pomoći
					
				\end{packed_enum}
			
				\item \underline{Vatrogasac(Spasioc) može:}
				
				\begin{packed_enum}
					
					\item biti osposobljen za vožnju autocisterni, autoljestvi, zapovjednog vozila i šumskog vozila
					
				\end{packed_enum}
			
				\item \underline{Policajac(Spasioc) može:}
				
				\begin{packed_enum}
					
					\item kretati se kao kontaktni policajac, pomoću motocikla, automobila ili oklopnog vozila
					
				\end{packed_enum}
			
				\item \underline{Voditelj stanice može:}
				
				\begin{packed_enum}
					
					\item definirati koji su spasioci dio njegove stanice
					\item definirati na koji način su spasioci iz njegove stanice osposobljeni voditi spašavanje
					\item odazvati se na akciju spašavanja
					
				\end{packed_enum}
			\item \underline{Dispečer može:}
			
			\begin{packed_enum}
				
				\item na temelju prijave otvarati akcije spašavanja s dostupnim informacijama i fotografijama
				\item vidjeti broj dostupnih spasioca po stanicama
				\item poslati zahtjev za uključivanjem spasilaca u akciju spašavanja
				\item prilikom slanja zahtjeva definirati na koji način bi spasilac trebao sudjelovati (auto, pješke.. )
				\item definirati razinu hitnosti zahtjeva
				\item ako je potrebno spasioca ukloniti s akcije
				\item ako je akcija spašavanja završila, označiti ju u sustavu kao gotovom
				\item preko karte spasiocima pojedinačno zadati zadatke
				\item pristupiti trenutnim pozicijama svih spasilaca zajedno s prikazom Voronojevog dijagrama
				\item odabrati da se za izradu dijagrama koriste pozicije svih spasioca, ili svih dostupnih neaktivnih spasioca, ili aktivnih spasioca na određenoj akciji
				
			\end{packed_enum}
			\item \underline{Administrator može:}
			
			\begin{packed_enum}
				
				\item vidjeti popis svih registriranih korisnika i njihovih osobnih podataka
				\item mijenjati dodijeljena prava, osobne podatke i pripadnost stanici
				
			\end{packed_enum}
			\end{packed_enum}
			
			\eject 
			
			
				
			\subsection{Obrasci uporabe}
				
				\textbf{\textit{dio 1. revizije}}
				
				\subsubsection{Opis obrazaca uporabe}
					\textit{Funkcionalne zahtjeve razraditi u obliku obrazaca uporabe. Svaki obrazac je potrebno razraditi prema donjem predlošku. Ukoliko u nekom koraku može doći do odstupanja, potrebno je to odstupanje opisati i po mogućnosti ponuditi rješenje kojim bi se tijek obrasca vratio na osnovni tijek.}\\
					

					\noindent \underbar{\textbf{UC$<$broj obrasca$>$ -$<$ime obrasca$>$}}
					\begin{packed_item}
	
						\item \textbf{Glavni sudionik: }$<$sudionik$>$
						\item  \textbf{Cilj:} $<$cilj$>$
						\item  \textbf{Sudionici:} $<$sudionici$>$
						\item  \textbf{Preduvjet:} $<$preduvjet$>$
						\item  \textbf{Opis osnovnog tijeka:}
						
						\item[] \begin{packed_enum}
	
							\item $<$opis korak jedan$>$
							\item $<$opis korak dva$>$
							\item $<$opis korak tri$>$
							\item $<$opis korak četiri$>$
							\item $<$opis korak pet$>$
						\end{packed_enum}
						
						\item  \textbf{Opis mogućih odstupanja:}
						
						\item[] \begin{packed_item}
	
							\item[2.a] $<$opis mogućeg scenarija odstupanja u koraku 2$>$
							\item[] \begin{packed_enum}
								
								\item $<$opis rješenja mogućeg scenarija korak 1$>$
								\item $<$opis rješenja mogućeg scenarija korak 2$>$
								
							\end{packed_enum}
							\item[2.b] $<$opis mogućeg scenarija odstupanja u koraku 2$>$
							\item[3.a] $<$opis mogućeg scenarija odstupanja  u koraku 3$>$
							
						\end{packed_item}
					\end{packed_item}
				
					
				\subsubsection{Dijagrami obrazaca uporabe}
					
					\textit{Prikazati odnos aktora i obrazaca uporabe odgovarajućim UML dijagramom. Nije nužno nacrtati sve na jednom dijagramu. Modelirati po razinama apstrakcije i skupovima srodnih funkcionalnosti.}
				\eject		
				
			\subsection{Sekvencijski dijagrami}
				
				\textbf{\textit{dio 1. revizije}}\\
				
				\textit{Nacrtati sekvencijske dijagrame koji modeliraju najvažnije dijelove sustava (max. 4 dijagrama). Ukoliko postoji nedoumica oko odabira, razjasniti s asistentom. Uz svaki dijagram napisati detaljni opis dijagrama.}
				\eject
	
		\section{Ostali zahtjevi}
		
			\textbf{\textit{dio 1. revizije}}\\
		 
			 \textit{Nefunkcionalni zahtjevi i zahtjevi domene primjene dopunjuju funkcionalne zahtjeve. Oni opisuju \textbf{kako se sustav treba ponašati} i koja \textbf{ograničenja} treba poštivati (performanse, korisničko iskustvo, pouzdanost, standardi kvalitete, sigurnost...). Primjeri takvih zahtjeva u Vašem projektu mogu biti: podržani jezici korisničkog sučelja, vrijeme odziva, najveći mogući podržani broj korisnika, podržane web/mobilne platforme, razina zaštite (protokoli komunikacije, kriptiranje...)... Svaki takav zahtjev potrebno je navesti u jednoj ili dvije rečenice.}
			 
			 
			 
	