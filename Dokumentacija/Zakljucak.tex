\chapter{Zaključak i budući rad}
		

	Sve skupa zadatak je izrađen kvalitetno. Ne savršeno, ima mjesta za popravak ali tim je zadovoljan gotovim rezultatom. Kao tim smo većinom dobro surađivali. Početak projekta (pisanje dokumentacije) tekao je glatko. Prve probleme susreli smo pokušavajući isprogramirati generičke funkcionalnosti, sve je dobro radilo osim ulogiravanja korisnika. CORS greška se tada prvi puta pojavila i sve odtad ju nismo uspjeli rješiti. To je rezultiralo neuspjelim pokušajima deployanja aplikacije na javni poslužitelj (prvo smo isprobali HEROKU). Nekako smo zaobišli problem i uspjeli prijaviti korisnika bez obzira na CORS grešku. Za vrijeme međuispita nismo radili. Zato je bilo teško vratiti se natrag na posao odmah nakon završetka ispita i trebalo je cijelom timu dosta vremena da bi počeli opet kvalitetnije raditi. Oko Božića smo shvatili da nemamo još puno vremena za dovršiti aplikaciju, a da nam je još puno preostalo. No ipak praznici su bili i malo se radilo. Nakon završetka praznika malo pomalo pa je cijeli tim stisnuo i progurao nekako sve funkcionalnosti koje dotad nismo imali. 90 posto aplikacije smo uspjeli napraviti kako treba. \par
	Od funkcionalnosti dugo nam je trebalo da smislimo rješenje za spremanje slika na backend i ta komunikacija (slanje slika sa frontenda na backend i obrnuto). Nakon što smo shvatili da se slika može jednostavno slati kao string unutar JSON-a, trebalo nam je samo koji dan da implementiramo i tu funkconalnost. Velik zalogaj je također bila mapa. Ali nakon što smo uspjeli postaviti mapu da se prikazuje, nije bilo većih problema sa prikazivanjem stvari na mapi. \par
	Nismo našli kvalitetno rješenje za prikazivanje i postavljanje komentara. Teško je bilo postaviti pop-up gdje bi korisnik napisao komentar. Imali smo i problema sa prikazivanjem voronijevih dijagrama, ali smo uspjeli prikazati lokacije korisnika i zadano ih filtrirati (bez dijagrama). \par
	Na backendu nije bilo previše „tehničkih“ poteškoća, više su bile poteškoće organizacije. Pošto je većina rađena u žurbi, nije bilo puno vremena za organizirati adrese za GET i POST. To je rezultiralo teško snalaženje frontend tima i time malo manje kvalitetan rad nego li je mogao biti. Definitivno je bilo puno truda uloženo u cjelokupno programiranje projekta, a time je stečeno i mnogo novih znanja (a i prijateljstava). Na frontendu se najviše naučilo o asinkronosti jezika Javascript i kako pametno manevrirati kroz HTML bez da javascript baci error zbog ne dohvaćanja stavki sa backenda. Na backendu su se naučile osnove rađenja unutar programskog okvira (JavaBeans). Moglo se više toga naučiti o samoj komunikaciji sa frontendom (kako funkcioniraju protokoli i kako dobro organizirati adrese za bolje i sigurnije snalaženje). Mislim da nam je svima nedostajalo znanja o timskom radu u kodiranju, što prvenstveno uključuje uredno i organizirano pisanje kodova (tako da ne može samo autor znati kako bi kod trebao raditi, nego i bilo tko drugi koji čita kod nakon njega). \par
	Kao voditelj tima, također mogu reći da mi je nedostajalo voditeljskog iskustva i da sam puno naučio o vođenju tima. Ima definitivno stvari koje bih učinio drugačije od početka da sam vodim ovaj projekt ispočetka. Na kraju je cijeli tim odlično radio na projektu i prije nastavka bismo definitivno uredili cijeli kod da radi kvalitetnije sad kada razumijemo neke nove stvari. \par

		
		\eject 